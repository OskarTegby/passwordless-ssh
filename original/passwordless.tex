\documentclass[10pt]{article}
\usepackage[a4paper, margin=1in]{geometry}
\usepackage[utf8]{inputenc}
\usepackage{hyperref}
\usepackage{microtype}


\title{Password-less SSH and \\ Using Visual Studio Code with UPPMAX}
\author{Oskar Tegby\thanks{Student of VT22}}
\date{7 February 2022}

\begin{document}
\maketitle

\section{Part 1 - Using SSH and SCP without entering your password}
This approach will work on MacOS and Linux based systems. This might work with Windows Subsystem for Linux
(WSL), however, we haven't tested this yet. Finally, please refer to manuals online if you are using
\href{https://devops.ionos.com/tutorials/use-ssh-keys-with-putty-on-windows/}{PuTTY}, or
\href{https://ubccr.freshdesk.com/support/solutions/articles/13000034093-using-ssh-keys-with-mobaxterm-windows-}{MobaXterm}.

\begin{enumerate}
\item Open Terminal.
\item Enter: \texttt{ssh-keygen -b 3072 -N "" -t rsa}
\item Enter: \texttt{ssh-copy-id <username>@rackham.uppmax.uu.se}\footnote{replace
\texttt{<username>} with your
UPPMAX username}
\end{enumerate}

Step 2 will create a public and private key in the default directory \texttt{\$HOME/.ssh/} with the
filenames \texttt{id\_rsa.pub} and \texttt{id\_rsa}. Be sure to keep the private key (\texttt{id\_rsa})
private - never share it to anyone else. The public key \texttt{id\_rsa.pub} is safe to share
around. Furthermore, in this command we are creating a public/private key pair using the RSA
algorithm with 3072 bits. 

\textbf{Advanced: }For this purpose we are setting the password to empty password. This is
convenient because users don't need to type in any password. However, for security purposes it is
considered safe to encrypt the key with a password. If you do want to encrypt your private key, you
can remove the \texttt{-N ""} option. If you have password-protected private keys, you can use
\texttt{ssh-agent} to cache your decrypted private key for the lifetime of your login session (no
password required while login session is alive).

%\noindent \textbf{Comment:} If step 3 fails, then you can try commenting out all of the lines in .bashrc on Rackham by
%logging in and entering "nano .bashrc" and commenting out the lines by inserting a # in the
%beginning of every non-empty line. When the installation is done, you can remove those comments.
%This just prevents that the ssh-copy-id script runs into issues.

\textbf{Result:} Now, you should be able to log into Rackham without entering any password by simply entering
\texttt{ssh <username>@rackham.uppmax.uu.se} in the terminal and pressing enter. Moreover, using this key pair,
you can also use password less \href{https://docs.github.com/en/authentication/connecting-to-github-with-ssh/adding-a-new-ssh-key-to-your-github-account}
{Github push and pull}.
This also works for the Linux servers that the university provides, and generally most Unix-based
machines (including Linux, and MacOS).

\section{Part 2 - Developing on Rackham from Visual Studio Code locally}
\begin{enumerate}
\item Download and install Visual Studio Code.
\item The extension “Remote - SSH” should be pre-installed, but to ensure that this is the case, press
\texttt{Cmd+Shift+X} or \texttt{Ctrl+Shift+X} in order to open the extensions pane. Enter “Remote - SSH” in the search box, and
ensure that it is there. Otherwise, simply install it by pressing the “install” button.
\item Press \texttt{Cmd/Ctrl+Shift+P}, type “Remote-SSH: Connect to Host” into the search box, and press enter.
\item Click “+Add New SSH Host”, type “ssh <username>@rackham.uppmax.uu.se”, and press enter.
\end{enumerate}

\textbf{Comment:} You should not need to enter the password if you followed the steps of part 1. If you have
to enter the password then do it and check if it works automatically afterwards.

\textbf{Result:} You should now be able to connect to the server and develop there using the integrated
terminal window in Visual Studio Code. The editor will allow you to open/read/edit files stored on
the remote server. You can also open files using the “Open Folder” button. That
way, you can navigate the file system on Rackham as well as test, build, and debug the code on that
system. 

\section{Acknowledgements}
\textbf{Oskar:} These instructions are a rewritten version of an
\href{https://stackoverflow.com/a/69970152/754562}{answer on Stack-overflow} and feedback from Chang
Hyun, which greatly reduced the complexity of the instructions and added further detail. Thanks
Chang Hyun!

\noindent\textbf{Chang Hyun:} Password-less SSH is one of the most convenient things that a person working on
Unix systems should set up. I thank Oskar for taking the initiative of writing up the information
to share with his classmates and future students of this course! I did make some changes/typesetting
and additions.

\end{document}
